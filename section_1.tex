\section{مقدمه}

دوران الگوریتمیک یا \lr{Algorithmic age} منحصر به زمان کنونی و آینده جهان است،
به گونه‌ای که الگوریتم‌های گوناگونی در زندگی انسان‌ها وارد شده‌اند. از ساده‌ترین
عملیات گرفته تا پیچیده‌ترین تصمیم‌گیری‌ها را می‌توان پیاده‌سازی کرد. این
الگوریتم‌ها حوزه‌های مختلف از قبیل اقتصاد، بهداشت، حمل و نقل و غیره را تحت تاثیر
خود قرار داده است. از انتخاب کوتاه‌ترین مسیر رانندگی گرفته تا پیشبینی و تشخیص
سرطان و طراحی پروتئین و دارو‌های مختلف.

این برگه به روشنی از قابلیت‌های این سیستم‌ها، که تصمیم‌گیری‌های مختلف را به
گونه‌ای انجام می‌دهند که کاربر نمی‌داند که این پیشبینی چگونه رخ داده است یا این
ماشین با چه داده‌هایی به چنین نتیجه‌ای رسیده است را \lr{Black box systems}
می‌نامد. زمانی که کاربر نمی‌تواند متوجه شود که پشت صحنه این پیشبینی‌ها به چه
شکلی می‌باشد می‌گوییم که این سیستم‌ها فاقد شفافیت \footnote{\lr{Transparency}}
هستند. فاقد شفافیت نگرانی‌هایی را در رابطه با پارامتر‌هایی مانند مسئولیت‌پذیری،
انصاف و پیامد‌های اخلاقی ایجاد می‌کند به ویژه در حوزه‌هایی که سیستم تصمیماتی را
می‌گیرد که می‌تواند روی مردم و جامعه به طور کلی تاثیر گذار باشد.

بر همین اساس، این برگه بحث‌هایی را در مورد شفافیت و اخلاق سیستم‌های مدرن مطرح
کرده است. درک چگونگی ادغام این نگرانی‌ها در سیستم‌ها و در نتیجه نحوه برخورد با
آنها در طول مهندسی نرم‌افزار و مهندسی نیازمندی‌ها بسیار مهم است.

\section{رویکرد تبیین‌پذیری}

رویکر و روش‌شناسی‌ای که این مقاله در مورد آن صحبت می‌کند تببین‌پذیری در
سیستم‌های نرم‌افزاری و حتی مدل‌های هوش مصنوعی است تا بتواند ضعف عدم شفافیت
سیستم‌ها را رفع کند.

\subsection{تببین‌پذیری چیست؟}

تببین‌پذیری یک روش مفید است تا از نگرانی‌های اخلاقی نرم‌افزار‌ها و مدل‌ها بکاهد.
به معنای قابلیت شرح نرم‌افزار و سیستم است. وقتی یک سیستم یا مدل هوش مصنوعی
تبیین‌پذیر است، به این معناست که عملکرد و تصمیمات آن قابل تفسیر و توجیه است. به
عبارت دیگر، می‌توان به راحتی فهمید که یک سیستم به چه شکلی کار می‌کند و چگونه به
تصمیمات خود رسیده است. تبیین‌پذیری یک ویژگی بسیار مهم در سیستم‌های نرم‌افزاری
است که موجب افزایش اعتماد به آن می‌شود و ارزش‌های اخلاقی و قانونی را در رابطه با
سیستم تعریف خواهد کرد. امروزه به مسئله تبیین‌پذیری سیستم‌ها بسیار اهمیت داده
می‌شود و یکی از مهم‌ترین نیازمندی‌های \lr{Non-functional} محسوب می‌شود. در حالتی
که به کاربران این اجازه را می‌دهد که خودشان بتوانند انتخاب کنند که از این سیستم
استفاده کنند یا از آن دوری کنند چرا که بر روی رابطه قابلیت اعتماد و اتکای سیستم
بسیار تاثیرگذار می‌باشد.

نکته: با توجه به قدرت هوش مصنوعی در تمام حوزه‌های زندگی بشر، تبیین‌پذیری به
عنوان یکی از مهم‌ترین پایه‌های اعتماد در نیازمندی‌های نرم‌افزار می‌باشد.


همچنین در این مقاله در مورد رابطه بین جنبه‌های کیفی و تبیین‌پذیری صحبت می‌شود.

\subsection{چالش‌های تبیین‌پذیری}

دلایل زیر نشان‌دهنده آن است که جمع‌آوری و استخراج داده، مذاکره و اعتبارسنجی در
فرایند تبیین‌پذیری با چالش‌هایی رو به رو می‌باشد:

\subsubsection{پیچیدگی سیستم‌ها}

در سیستم‌هایی که مبنی بر هوش مصنوعی و فرایند یادگیری ماشین هستند با وجود
الگوریتم‌های مختلف که وظیفه تصمیم‌گیری را در سیستم دارند، سطح پیچیدگی بسیار بالا
می‌باشد. درک و توضیح این سیستم‌ها با فرایند‌هایشان برای کاربران مختلف به مفهوم
ساده، بسیار سخت و غیرقابل درک می‌باشد.

\subsubsection{طبعیت \lr{Black box}}

از نظر کاربران، بسیاری از الگوریتم‌ها به شکل جادویی عمل می‌کنند، بدان معنا که
فرایند‌های داخلی این الگوریتم‌ها کاملا به صورت مات می‌باشد و توسط انسان بدون
دانش قبلی به راحتی قابل درک نیست.

\subsubsection{زمینه‌گرایی توضیح یا \lr{Subjectivity of Explanation}}

زمینه‌گرایی توضیح به معنای نسبی بودن یا وابستگی توضیحات به نگرش و دیدگاه فردی
است. در حالت کلی تفسیر هر چیزی توسط ذینفعان می‌تواند کاملا متفاوت از نظر معنا و
دیدگاه باشد. مذاکره برای به اجماع رسیدن در سطح و نوع توضیح مورد نیاز می‌تواند
چالش برانگیز باشد، به ویژه زمانی که با دیدگاها و علایق گوناگون سروکار داریم.

\subsubsection{تریدآف همراه با تاثیرگذاری روی عملکرد یا \lr{Trade-off with
Performance}}

گاهی افزایش تبیین‌پذیری در یک سیستم می‌تواند به قیمت عملکرد و کارایی تمام شود.
یک مهندس نیازمندی باید بتواند بین تبیین‌پذیری با سایر الزامات سیستم \lr{System
requirements} تعادل  ایجاد کند.