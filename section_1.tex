\section{رویکرد و راه‌حل}

رویکر و روش‌شناسی‌ای که این مقاله در مورد آن صحبت می‌کند تببین‌پذیری در
سیستم‌های نرم‌افزاری و حتی مدل‌های هوش مصنوعی است تا بتواند ضعف عدم شفافیت
سیستم‌ها را رفع کند.

\subsection{تببین‌پذیری چیست؟}

تببین‌پذیری یک روش مفید است تا از نگرانی‌های اخلاقی نرم‌افزار‌ها و مدل‌ها بکاهد.
به معنای قابلیت شرح نرم‌افزار و سیستم است. وقتی یک سیستم یا مدل هوش مصنوعی
تبیین‌پذیر است، به این معناست که عملکرد و تصمیمات آن قابل تفسیر و توجیه است. به
عبارت دیگر، می‌توان به راحتی فهمید که یک سیستم به چه شکلی کار می‌کند و چگونه به
تصمیمات خود رسیده است. تبیین‌پذیری یک ویژگی بسیار مهم در سیستم‌های نرم‌افزاری
است که موجب افزایش اعتماد به آن می‌شود و ارزش‌های اخلاقی و قانونی را در رابطه با
سیستم تعریف خواهد کرد. امروزه به مسئله تبیین‌پذیری سیستم‌ها بسیار اهمیت داده
می‌شود و یکی از مهم‌ترین نیازمندی‌های \lr{Non-functional} محسوب می‌شود. در حالتی
که به کاربران این اجازه را می‌دهد که خودشان بتوانند انتخاب کنند که از این سیستم
استفاده کنند یا از آن دوری کنند چرا که بر روی رابطه قابلیت اعتماد و اتکای سیستم
بسیار تاثیرگذار می‌باشد.

نکته: با توجه به قدرت هوش مصنوعی در تمام حوزه‌های زندگی بشر، تبیین‌پذیری به
عنوان یکی از مهم‌ترین پایه‌های اعتماد در نیازمندی‌های نرم‌افزار می‌باشد.


همچنین در این مقاله در مورد رابطه بین جنبه‌های کیفی و تبیین‌پذیری صحبت می‌شود.

\subsection{چالش‌های تبیین‌پذیری}

دلایل زیر نشان‌دهنده آن است که جمع‌آوری و استخراج داده، مذاکره و اعتبارسنجی در
فرایند تبیین‌پذیری با چالش‌هایی رو به رو می‌باشد:

\subsubsection{پیچیدگی سیستم‌ها}

در سیستم‌هایی که مبنی بر هوش مصنوعی و فرایند یادگیری ماشین هستند با وجود
الگوریتم‌های مختلف که وظیفه تصمیم‌گیری را در سیستم دارند، سطح پیچیدگی بسیار بالا
می‌باشد. درک و توضیح این سیستم‌ها با فرایند‌هایشان برای کاربران مختلف به مفهوم
ساده، بسیار سخت و غیرقابل درک می‌باشد.

\subsubsection{طبعیت \lr{Black box}}

از نظر کاربران، بسیاری از الگوریتم‌ها به شکل جادویی عمل می‌کنند، بدان معنا که
فرایند‌های داخلی این الگوریتم‌ها کاملا به صورت مات می‌باشد و توسط انسان بدون
دانش قبلی به راحتی قابل درک نیست.

\subsubsection{زمینه‌گرایی توضیح یا \lr{Subjectivity of Explanation}}

زمینه‌گرایی توضیح به معنای نسبی بودن یا وابستگی توضیحات به نگرش و دیدگاه فردی
است. در حالت کلی تفسیر هر چیزی توسط ذینفعان می‌تواند کاملا متفاوت از نظر معنا و
دیدگاه باشد. مذاکره برای به اجماع رسیدن در سطح و نوع توضیح مورد نیاز می‌تواند
چالش برانگیز باشد، به ویژه زمانی که با دیدگاها و علایق گوناگون سروکار داریم.

\subsubsection*{عدم درک مشترک \footnote{\lr{Lack of shared understanding}}}

یکی دیگر از دشواری‌ها، ارتباطات مناسب در مهندسی نیازمندی است. ذینفعان بیرونی و
تیم توسعه ممکن است ناخواسته از یکسری کلمات متفاوت با مفهوم یکسان استفاده کنند که
در نهایت باعث ایجاد سوء‌تفاهم و نقص فهم مشترک بین افراد شود که در نهایت چالشی
برای ارتباط با یکدیگر ایجاد می‌کند. لازمه کارآمدی ارتباطات درک مشترک از مفاهیم
می‌باشد که ریسک دوباره‌کاری و نارضایتی ذینفعان را کاهش می‌دهد.

\subsubsection*{رویکرد درک مشترک بین افراد}

مهندسان نرم‌افزار می‌توانند مجموعه‌ای از فرآورده‌ها را ایجاد کنند که باعث ایجاد
درک و فهم مشترک در پروژه‌های نرم‌افزاری می‌شود و بار‌ها قابل استفاده مجدد و
اصلاح خواهند بود تا فرآورده‌ها، محصولی از مذاکره با زبانی مشترک بین افراد باشد.

\subsubsection*{فرآورده‌ها}

فرآورده‌ها هر گونه اسناد متنی و اشکال گرافیکی هستند که به دور از کد‌ها و
محصولاتی نرم‌افزاری، ابزاری برای مذاکره بین تمام افراد‌ حاضر (چه ذینفعان چه
مهندسان مختلف) می‌باشند. محتوای فرآورده‌ها معمولا اشکال، متن‌ها، مدل‌های بصری،
فهرست‌ها، چارت‌ها ، چهارچوب‌ها و مدل‌های کیفیت می‌باشد. این فرآورده‌ها در
شکل‌دهی ساختار پروژه بسیار کارآمد هستند به گونه‌ای که در فرایند‌های مهندسی
نیازمندی از قبیل، مدل‌های مفهومی \footnote{\lr{Conceptual models}} کاتالوگ دانش
\footnote{\lr{Knowledge catelogues}} و مدل‌های مرجع \footnote{\lr{Reference
models}} کاربرد متعددی دارند.

\subsubsection{تریدآف همراه با تاثیرگذاری روی عملکرد یا \lr{Trade-off with
Performance}}

گاهی افزایش تبیین‌پذیری در یک سیستم می‌تواند به قیمت عملکرد و کارایی تمام شود.
یک مهندس نیازمندی باید بتواند بین تبیین‌پذیری با سایر الزامات سیستم \lr{System
requirements} تعادل  ایجاد کند. برای درک این چالش مثال زیر را مطالعه کنید:

تصور کنید یک شرکت در حال توسعه سیستم توصیه‌گرا برای اپلیکیشن تجاری خود می‌باشد.
این سیستم الگوریتم‌های پیچیده \lr{ML} را برای تحلیل رفتار‌ها و ترجیحات
\footnote{\lr{Preferences}} کاربران استفاده می‌کند تا بتواند محصولات مشابه
علاقه‌مندی آنها را به نحوی معرفی کند که کاربران انتظار داشتند. یکی از
نیازمندی‌های \lr{NFR} این سیستم، ارائه توضیحات برای هر کدام از نتایج محصولات
توصیه شده می‌باشد تا بتواند موجب اعتماد و رضایت کاربران شود. در این صورت گنجاندن
توضیحات به همراه جزئیات چرایی انتخاب این مورد (محصول) به عنوان مورد مرتبط برای
این سیستم تاثیر به سزایی در عملکرد آن خواهد داشت. این عمل باعث تاخیری در تولید
این موارد برای کاربران می‌شود که از نظر تجربه کاربری \footnote{\lr{User
experience}} یک ضعف محسوب می‌شود به ویژه زمانی که کاربران انتظار دارند که تمام
تقاضا‌هایشان از سیستم در کمتر از پنج ثانیه پاسخ داده شود.

احتمالاً برای این مثال راهکار‌های زیر در نظر گرفته می‌شود تا ضمن تبیین‌پذیری
سیستم، عملکرد سیستم نیز مانند سابق با سرعت بالا حفظ شود:

\begin{enumerate}
    \item کاهش پیچیدگی توضیحات: به جای آنکه توضیحات کاملی در مورد عملکرد
    الگوریتم‌های هوش مصنوعی به ازای هر مورد فراهم شود، سیستم می‌تواند بسیار ساده
    با ارائه خلاصه‌ای مفید، فاکتور‌های مهم و اساسی دلیل انتخاب موارد به عنوان
    توصیه کاربر را مشخص کند.
    \item استفاده از متد‌های فنی در مهندسی نرم‌افزار مانند فرایند‌های کش کردن
    انتخاب‌های کاربر (براساس کلیک‌های مختلف روی محصولات یا مدت زمانی که روی
    محصول مورد نظر کاربر مطالعه داشته) محاسبات از پیش تعیین شده‌ای در مورد چرایی
    انتخاب محصول به عنوان توصیه را مشخص کند.
\end{enumerate}

انتخاب استراتژی مناسب برای حفظ تبیین‌پذیری به همراه سرعت و کارایی بالا در عملکرد
سیستم توصیه‌گرا، یک تریدآفی است که وظیفه آن بر عهده تیم توسعه، طراح و معماری
نرم‌افزار می‌باشد.

\subsubsection{سیستم‌های پویا و در حال تکامل}

یکی از چالش‌های مهم تبیین‌پذیری سیستم‌هایی است که در طول زمان دچار تغییرات کلی
به ویژه در نیازمندی‌ها می‌شوند. اینکه توضیحات متناسب با تغییر سیستم‌ها به روز
شود چالشی مهم است.

\subsubsection{اعتبارسنجی و اعتماد}

اعتبار بخشیدن به توضیحات ارائه شده توسط یک سیستم می تواند دشوار باشد، به ویژه
زمانی که آنها شامل فرآیندها یا داده های پیچیده باشند. ایجاد اعتماد در این
توضیحات مستلزم روش‌های اعتبارسنجی قوی و شفافیت در فرآیند تولید توضیح است.

همچنین اشاره می‌کند که مهندسی نیازمندی فرایند ساده برای شناسایی و مشخص کردن
نیازمندی‌ها نیست، بلکه فرایندی جهت حمایت از ارتباطات کارآمد این نیازمندی‌ها بین
ذینفعان مختلف می‌باشد.
