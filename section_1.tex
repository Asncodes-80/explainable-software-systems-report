\section{رویکرد و راه‌حل}

رویکر و روش‌شناسی‌ای که این مقاله در مورد آن صحبت می‌کند تببین‌پذیری در
سیستم‌های نرم‌افزاری و حتی مدل‌های هوش مصنوعی است تا بتواند ضعف عدم شفافیت
سیستم‌ها را رفع کند.

\subsection{تببین‌پذیری چیست؟}

تببین‌پذیری یک روش مفید است تا از نگرانی‌های اخلاقی نرم‌افزار‌ها و مدل‌ها بکاهد.
به معنای قابلیت شرح نرم‌افزار و سیستم است. وقتی یک سیستم یا مدل هوش مصنوعی
تبیین‌پذیر است، به این معناست که عملکرد و تصمیمات آن قابل تفسیر و توجیه است. به
عبارت دیگر، می‌توان به راحتی فهمید که یک سیستم به چه شکلی کار می‌کند و چگونه به
تصمیمات خود رسیده است. تبیین‌پذیری یک ویژگی بسیار مهم در سیستم‌های نرم‌افزاری
است که موجب افزایش اعتماد به آن می‌شود و ارزش‌های اخلاقی و قانونی را در رابطه با
سیستم تعریف خواهد کرد. امروزه به مسئله تبیین‌پذیری سیستم‌ها بسیار اهمیت داده
می‌شود و یکی از مهم‌ترین نیازمندی‌های \lr{Non-functional} محسوب می‌شود. در حالتی
که به کاربران این اجازه را می‌دهد که خودشان بتوانند انتخاب کنند که از این سیستم
استفاده کنند یا از آن دوری کنند چرا که بر روی رابطه قابلیت اعتماد و اتکای سیستم
بسیار تاثیرگذار می‌باشد.

نکته: با توجه به قدرت هوش مصنوعی در تمام حوزه‌های زندگی بشر، تبیین‌پذیری به
عنوان یکی از مهم‌ترین پایه‌های اعتماد در نیازمندی‌های نرم‌افزار می‌باشد.

همچنین در این مقاله در مورد رابطه بین جنبه‌های کیفی و تبیین‌پذیری صحبت می‌شود.

\subsection{چالش‌های تبیین‌پذیری}

دلایل زیر نشان‌دهنده آن است که جمع‌آوری و استخراج داده، مذاکره و اعتبارسنجی در
فرایند تبیین‌پذیری با چالش‌هایی رو به رو می‌باشد:

\subsubsection{پیچیدگی سیستم‌ها}

در سیستم‌هایی که مبنی بر هوش مصنوعی و فرایند یادگیری ماشین هستند با وجود
الگوریتم‌های مختلف که وظیفه تصمیم‌گیری را در سیستم دارند، سطح پیچیدگی بسیار بالا
می‌باشد. درک و توضیح این سیستم‌ها با فرایند‌هایشان برای کاربران مختلف به مفهوم
ساده، بسیار سخت و غیرقابل درک می‌باشد.

\subsubsection{طبعیت \lr{Black box}}

از نظر کاربران، بسیاری از الگوریتم‌ها به شکل جادویی عمل می‌کنند، بدان معنا که
فرایند‌های داخلی این الگوریتم‌ها کاملا به صورت مات می‌باشد و توسط انسان بدون
دانش قبلی به راحتی قابل درک نیست.

\subsubsection{زمینه‌گرایی توضیح یا \lr{Subjectivity of Explanation}}

زمینه‌گرایی توضیح به معنای نسبی بودن یا وابستگی توضیحات به نگرش و دیدگاه فردی
است. در حالت کلی تفسیر هر چیزی توسط ذینفعان می‌تواند کاملا متفاوت از نظر معنا و
دیدگاه باشد. مذاکره برای به اجماع رسیدن در سطح و نوع توضیح مورد نیاز می‌تواند
چالش برانگیز باشد، به ویژه زمانی که با دیدگاها و علایق گوناگون سروکار داریم.

\subsubsection*{عدم درک مشترک \footnote{\lr{Lack of shared understanding}}}

یکی دیگر از دشواری‌ها، ارتباطات مناسب در مهندسی نیازمندی است. ذینفعان بیرونی و
تیم توسعه ممکن است ناخواسته از یکسری کلمات متفاوت با مفهوم یکسان استفاده کنند که
در نهایت باعث ایجاد سوء‌تفاهم و نقص فهم مشترک بین افراد شود که در نهایت چالشی
برای ارتباط با یکدیگر ایجاد می‌کند. لازمه کارآمدی ارتباطات درک مشترک از مفاهیم
می‌باشد که ریسک دوباره‌کاری و نارضایتی ذینفعان را کاهش می‌دهد.

\subsubsection*{رویکرد درک مشترک بین افراد}

مهندسان نرم‌افزار می‌توانند مجموعه‌ای از فرآورده‌ها را ایجاد کنند که باعث ایجاد
درک و فهم مشترک در پروژه‌های نرم‌افزاری می‌شود و بار‌ها قابل استفاده مجدد و
اصلاح خواهند بود تا فرآورده‌ها، محصولی از مذاکره با زبانی مشترک بین افراد باشد.

\subsubsection*{فرآورده‌ها}

فرآورده‌ها هر گونه اسناد متنی و اشکال گرافیکی هستند که به دور از کد‌ها و
محصولاتی نرم‌افزاری، ابزاری برای مذاکره بین تمام افراد‌ حاضر (چه ذینفعان چه
مهندسان مختلف) می‌باشند. محتوای فرآورده‌ها معمولا اشکال، متن‌ها، مدل‌های بصری،
فهرست‌ها، چارت‌ها ، چهارچوب‌ها و مدل‌های کیفیت می‌باشد. این فرآورده‌ها در
شکل‌دهی ساختار پروژه بسیار کارآمد هستند به گونه‌ای که در فرایند‌های مهندسی
نیازمندی از قبیل، مدل‌های مفهومی \footnote{\lr{Conceptual models}} کاتالوگ دانش
\footnote{\lr{Knowledge catelogues}} و مدل‌های مرجع \footnote{\lr{Reference
models}} کاربرد متعددی دارند.

\subsubsection{تریدآف همراه با تاثیرگذاری روی عملکرد یا \lr{Trade-off with
Performance}}

گاهی افزایش تبیین‌پذیری در یک سیستم می‌تواند به قیمت عملکرد و کارایی تمام شود.
یک مهندس نیازمندی باید بتواند بین تبیین‌پذیری با سایر الزامات سیستم \lr{System
requirements} تعادل  ایجاد کند. برای درک این چالش مثال زیر را مطالعه کنید:

تصور کنید یک شرکت در حال توسعه سیستم توصیه‌گرا برای اپلیکیشن تجاری خود می‌باشد.
این سیستم الگوریتم‌های پیچیده \lr{ML} را برای تحلیل رفتار‌ها و ترجیحات
\footnote{\lr{Preferences}} کاربران استفاده می‌کند تا بتواند محصولات مشابه
علاقه‌مندی آنها را به نحوی معرفی کند که کاربران انتظار داشتند. یکی از
نیازمندی‌های \lr{NFR} این سیستم، ارائه توضیحات برای هر کدام از نتایج محصولات
توصیه شده می‌باشد تا بتواند موجب اعتماد و رضایت کاربران شود. در این صورت گنجاندن
توضیحات به همراه جزئیات چرایی انتخاب این مورد (محصول) به عنوان مورد مرتبط برای
این سیستم تاثیر به سزایی در عملکرد آن خواهد داشت. این عمل باعث تاخیری در تولید
این موارد برای کاربران می‌شود که از نظر تجربه کاربری \footnote{\lr{User
experience}} یک ضعف محسوب می‌شود به ویژه زمانی که کاربران انتظار دارند که تمام
تقاضا‌هایشان از سیستم در کمتر از پنج ثانیه پاسخ داده شود.

احتمالاً برای این مثال راهکار‌های زیر در نظر گرفته می‌شود تا ضمن تبیین‌پذیری
سیستم، عملکرد سیستم نیز مانند سابق با سرعت بالا حفظ شود:

\begin{enumerate}
    \item کاهش پیچیدگی توضیحات: به جای آنکه توضیحات کاملی در مورد عملکرد
    الگوریتم‌های هوش مصنوعی به ازای هر مورد فراهم شود، سیستم می‌تواند بسیار ساده
    با ارائه خلاصه‌ای مفید، فاکتور‌های مهم و اساسی دلیل انتخاب موارد به عنوان
    توصیه کاربر را مشخص کند.
    \item استفاده از متد‌های فنی در مهندسی نرم‌افزار مانند فرایند‌های کش کردن
    انتخاب‌های کاربر (براساس کلیک‌های مختلف روی محصولات یا مدت زمانی که روی
    محصول مورد نظر کاربر مطالعه داشته) محاسبات از پیش تعیین شده‌ای در مورد چرایی
    انتخاب محصول به عنوان توصیه را مشخص کند.
\end{enumerate}

انتخاب استراتژی مناسب برای حفظ تبیین‌پذیری به همراه سرعت و کارایی بالا در عملکرد
سیستم توصیه‌گرا، یک تریدآفی است که وظیفه آن بر عهده تیم توسعه، طراح و معماری
نرم‌افزار می‌باشد.

\subsubsection{سیستم‌های پویا و در حال تکامل}

یکی از چالش‌های مهم تبیین‌پذیری سیستم‌هایی است که در طول زمان دچار تغییرات کلی
به ویژه در نیازمندی‌ها می‌شوند. اینکه توضیحات متناسب با تغییر سیستم‌ها به روز
شود چالشی مهم است.

\subsubsection{اعتبارسنجی و اعتماد}

اعتبار بخشیدن به توضیحات ارائه شده توسط یک سیستم می تواند دشوار باشد، به ویژه
زمانی که آنها شامل فرآیندها یا داده های پیچیده باشند. ایجاد اعتماد در این
توضیحات مستلزم روش‌های اعتبارسنجی قوی و شفافیت در فرآیند تولید توضیح است.

همچنین اشاره می‌کند که مهندسی نیازمندی فرایند ساده برای شناسایی و مشخص کردن
نیازمندی‌ها نیست، بلکه فرایندی جهت حمایت از ارتباطات کارآمد این نیازمندی‌ها بین
ذینفعان مختلف می‌باشد.

\section{سابقه دانشی}

برای اجماع فهم مشترک بر مسائل مختلف این حوزه از مهندسی نرم‌افزار، خواننده نیاز
دارد که با مفاهیم زیر به صورت کلی آشنا باشد تا بتواند:

\begin{enumerate}
    \item از دانشی فراتر از زمینه‌های خود استفاده کنند و از این دانش برای رفع
    نیاز‌های یک پروژه خاص (جاری یا جدید) استفاده کنند.
    \item دستیابی به درکی مشترک که منجر به ارتباطات بهتر و تعریف نیازمندی‌های
    سیستم به شکل «درست» می‌شود.
\end{enumerate}

\subsection{تعاریف یا \lr{Definitions}}

تعاریف در \lr{SE} و \lr{RE}، راهنمایی تقریبی برای مهندسان نرم‌افزار در مورد
دامنه، عناصر و هر چیز دیگری را ارائه می‌دهند. به این صورت که یکی از مهم‌ترین
مراحل تسهیل ارتباطات برای یک موضوع یا یک مفهوم می‌باشند. وقتی در مورد تعریف
جنبه‌های کیفی یا \lr{Quality aspects} صحبت می‌شود در حقیقت منظور همان
راهنمایی‌ها برای مهندسان نرم‌افزار است که به آنها در فهمیدن روند مهندسی نیازمندی
به خصوص تضمین کیفیت یا \lr{Quality assurance} کمک می‌کند. عدم اجماع حاضرین بر سر
مفاهیم و تعاریف می‌تواند سبب ایجاد نتیجه نامناسبی در مشخصات و یکپارچگی
نیازمندی‌های غلط شود. برای مثال وقتی در جلسات در مورد \lr{Usability} صبحت
می‌شود، برخی از افراد توسعه دهنده منظور را در رابطه با استفاده و بهره‌وری بلند
مدت \footnote{\lr{Long-term efficiency}} می‌دانند و برخی از افراد منظور را در
استفاده آسان محصول \footnote{\lr{Ease of Use (EoU)}} می‌دانند.

\subsection{مدل‌ها یا \lr{Models}}

یک مدل در بالاترین سطح تجرید در مورد کارایی سیستم تمرکز دارد که بتواند تمام
جنبه‌های سیستم را به سادگی و به دور از جزئیات نمایش دهد. هیچ وقت یک مدل به
تنهایی، تمام سیستم را تشریح نمی‌کند. مدل‌ها در طیف گسترده‌ای از جنبه‌های مختلف
یک سیستم استفاده می‌شوند. مدل‌ها می‌توانند برای اهداف توسعه نرم‌افزار و توسعه
کسب و کار استفاده شوند. از نظر نرم‌افزاری می‌توانند نقش مهمی در تعریف ساختار
نرم‌افزار یا پیکربندی‌ها داشته باشند و از سوی دیگر برای توصیف و بهینه‌سازی
نگرانی‌های سازمانی مانند فرآیندها و حوزه‌های تجاری می‌توانند بسیار مفید باشند.
از انواع مدل‌ها می‌توان به موارد زیر اشاره کرد:

\subsubsection{مدل‌های مفهومی یا \lr{Conceptual models}}

مدل‌هایی هستند که برای تعریف و توصیف یک مفهوم استفاده می‌شوند. افراد را قادر
می‌سازند که بتوانند ساختار و ویژگی‌های یک جنبه کیفی مشخص را در طول فرآیند تحلیل
نیازمندی درک کنند. دانش مورد نیاز برای توسعه مدل‌های مفهومی معمولاً از
\lr{Literature}، تجارب قبلی مشابه و حوزه تخصصی استخراج می‌شوند.

\subsubsection{مدل‌های کیفی یا \lr{Quality models}}

مدل‌هایی که صفات کیفی و ویژگی‌های نیازمندی نرم‌افزار را تعریف و مشخص می‌کنند.
این مدل‌ها معمولاً روش‌های سیستماتیکی را برای ارزیابی و اطمینان کیفی نیازمندی‌ها
بر اساس استانداردی معتبر بکار می‌گیرند. این مدل‌ها به ذینفعان کمک می‌کنند تا درک
کنند که لازمه بالا بودن کیفیت خدمات چیست و دستورالعمل‌هایی را برای ایجاد، تجزیه
و تحلیل و اعتبارسنجی نیازمندی‌ها ارائه می‌دهد.

\begin{enumerate}
    \item استاندارد \lr{ISO/IEC 25010 (SQuaRE)}: یک استاندارد جامع برای
    نیازمندی‌ها و ارزیابی کیفیت نرم‌افزار است. ویژگی‌های مورد بررسی آن از قبیل،
    عملکرد، قابلیت استفاده، کارایی، قابلیت نگهداری و غیره می‌باشند.
    \item مدل \lr{Quality Function Deployment (QFD)}: رویکردی مشتری محور است که
    مهم‌ترین وظیفه آن ترجمه نیاز‌های مشتری به نیازمندی‌های محصول خاص است. این
    مدل به مهندسان مخصوصاً مهندسان نیازمندی کمک می‌کند تا بتوانند تطابق
    نیازمندی‌های پیاده‌سازی شده را با انتظارات و ترجیحات مشتریان بررسی کنند.
    \item \lr{McCall's Quality Model}: این مدل توسط جان.دی.مک‌کال مطرح شده است و
    ۱۱ عامل کیفی را که در سه دسته: عملکرد محصول، بازبینی محصول و جا به جایی و
    انتقال محصول قرار دارند، را شناسایی می‌کند. این عوامل شامل قابلیت اعتماد،
    قابلیت استفاده، کارایی و ویژگی‌های مشابه با استاندارد \lr{SQuaRE} هستند.
    \item \lr{IEEE 730 Standard}: این استاندارد فرآیند‌های اطمینان کیفیت را برای
    پروژه‌های توسعه نرم‌افزار از جمله مهندسی نیازمندی‌ها را تعریف می‌کند.
    فعالیت‌های مرتبط با برنامه‌ریزی کیفیت، اطمینان کیفیت و کنترل کیفیت را در طول
    چرخه‌ی عمر توسعه نرم‌افزار شامل می‌شود.
    \item \lr{CMMI (Capability Maturity Model Integration)}: یک چهارچوب برای
    بهبود فرآیند‌های مرتبط با توسعه و نگهداری سیستم‌های نرم‌افزاری را ارائه
    می‌دهد. فعالیت‌های این چهارچوب شامل روش‌های مرتبط با مدیریت نیازمندی‌ها
    می‌باشد و اهمیت مدیریت و اطمینان از کیفیت نیازمندی‌ها را در طول چرخه توسعه
    تاکید می‌کند. در این استاندارد تمام فرآیند‌های مختلف توسعه نرم‌افزار از
    پایه‌ای‌ترین سطح تا بالاترین سطح ارزیابی می‌شوند و امتیازی برای هر سطح
    متناسب با میزان تسلط و بهره‌وری به آنها اختصاص می‌یابد. به کمک این امتیازات
    و نمرات مشخص می‌شود که هر سطح به چه میزانی رشد و بهبود داشته است.
\end{enumerate}

\subsubsection*{منظور از \lr{Blueprint} در حوزه توسعه نرم‌افزار و چرخه
نیازمندی‌ها}

معمولاً جزئیات مشخصی را در مورد ساختار‌ها، طراحی بخش‌ها و فانکشنالیتی یک سیستم
نرم‌افزاری یا اپلیکیشن بیان می‌کند. این \lr{Blueprint} به برنامه نویسان و
توسعه‌دهندگان کمک می‌کند تا از آن به عنوان راهنما در حوزه فهم مناسب از اینکه چه
چیزی باید پیاده‌سازی شود، استفاده می‌کنند.

در مهندسی نیازمندی‌ها هم ممکن است مانند همین مفهوم به کار رود با این تفاوت که
نیازمندی‌های \lr{functional} و \lr{Non-functional} در کنار یکدیگر مطرح می‌شوند
تا معماری و ساختار نرم‌افزار را به وسیله مدل‌های داده محور طراحی و توسعه دهند.

\subsubsection{مد‌های مرجع یا \lr{Reference models}}

مدلی که شامل حداقل مقدار از مجموعه‌ای از مفاهیم، بدیهیات و روابط در یک دامنه
مسئله خاص می‌باشد و به هیچ یک از استاندارد‌ها ،فناوری‌ها، کد‌های نوشته شده،
اجراها یا سایر جزئیات وابستگی ندارد. مدل‌های مرجع می‌توانند به صورت
\lr{Blueprint} در مهندسی نرم‌افزار استفاده شوند تا بتوانند زیرساخت نرم‌افزاری را
ارائه دهند. نام‌های مختلفی را به مدل‌های مرجع اختصاص دادند مانند نام‌های زیر:

\begin{itemize}
    \item \lr{Universal models}
    \item \lr{Generic models}
    \item \lr{Model patterns}
\end{itemize}

برای نمونه، جهت درک بهتر پروتکل‌های شبکه و نحوه ارسال و دریافت داده‌ها، مدلی هفت
لایه نام \footnote{\lr{Open System Interconnection}} \lr{OSI} را معرفی کردند که
در بالاترین‌ سطح تجرید می‌توان تعاریف هر لایه به همراه وظایف آن‌ها را به بهترین
شکل آموزش داد. مدل مرجع \lr{OSI} به وسیله مهندسان شبکه برای توصیف معماری‌های
شبکه مورد استفاده قرار می‌گیرد تا بتوانند متناسب با پروتکل‌ها برنامه‌های مورد
نیاز خود را توسعه دهند.

مهم‌ترین دلیل استفاده از مدل‌های مرجع را در زیر به دو شکل بیان کرده‌ایم:

\begin{enumerate}
    \item یک چهارچوب یا نمونه‌ای با سطح بالای تجرید، جهت درک کامل روابط میان
    موجودیت‌ها در یک محیط یا دامنه است (مانند شبکه‌های کامپیوتری یا سیستم‌های
    تبیین پذیر)
    \item جهت استانداردسازی یا توصیف فرایند‌های توسعه. ذات این نوع سطح از تجرید
    به مهندسان انعطاف‌پذیری را اعطا می‌کند که می‌توانند در موقعیت‌های مختلف به
    راحتی سازگار شوند.
\end{enumerate}


\subsection{کاتالوگ‌ها یا \lr{Catalogues}}

کاتالوگ دانش مجموعه‌ای سازمان‌دهی شده از منابع دانشی است که درون یک سازمان وجود
دارد. این منابع می‌توانند شامل انواع دانش‌ها مانند اسناد، گزارش‌ها، روش‌های
توسعه و بهترین رویکرد‌های حل مسئله، مواد آموزشی و موارد دیگر باشند. هدف اصلی از
کاتالوگ‌ها تسهیل در توسعه، به اشتراک‌گذاری و استفاده مجدد از منابع دانش در یک
سازمان در پروژه‌های مشابه می‌باشد. بعضی از محققان کاتالوگی را برای دامنه مشخصی
مبتنی بر فرضیه چهارچوب‌های \lr{NFR} توسعه داده‌اند به گونه‌ای که نتیجه این توسعه
می‌تواند به تریدآف "چگونه یک یا چند \lr{NFR} در یک سیستم رابطه و تعامل دارند و
چگونه می‌توانند با یکدیگر همزستی داشته باشند" بپردازد. از نمونه‌های این
کاتالوگ‌ها می‌توان به موارد زیر اشاره کرد:

\begin{itemize}
    \item \lr{Serrano and Serrano} یک کاتالوگ به طور خاص برای دامنه محاسباتی
    فراگیر و سیار ایجاد کرده‌اند \cite{serrano2013ubiquitous}.
    \item \lr{Torres and Martins} پیشنهاد استفاده از کاتالوگ‌های \lr{NFR} را در
    ساخت برنامه‌های میان‌افزاری \lr{RFID} برای کاهش چالش‌های استخراج داده
    \lr{NFR} در سیستم‌های مستقل، را مطرح کرده‌اند \cite{torres2014nfr}.
\end{itemize}

تمام این نمونه‌ها سعی در این داشتند که کاتالوگ‌ها ابزاری برای کاهش و حذف خطا‌های
احتمالی در شناسایی \lr{FR}ها و \lr{NFR}ها باشند.